\usepackage{sidenotes} % Required
\usepackage{multirow} % Required
\usepackage{diagbox} % Used for boolean algorithms.

% Command to consistently format side-panel with C++ versions that introduced different variants of the algorithm.
\newcommand{\cppversions}[5]{
\begin{margintable}
\footnotesize
\begin{tabular}{|m{\dimexpr.5\marginparwidth-2\tabcolsep-1pt}|m{\dimexpr.5\marginparwidth-2\tabcolsep-1pt}|}
\rowcolor{black!80} \multicolumn{2}{c}{\textcolor{white}{#1}} \\
introduced & #2 \\
\hline
constexpr & #3 \\
\hline
parallel & #4 \\
\hline
rangified & #5 \\
\hline
\end{tabular}
\end{margintable}
}

% Command to consistently format the algorithm info table.
\newcommand{\constraints}[4]{
\begin{center}
\footnotesize
\begin{tabular}{|m{\dimexpr.22\textwidth-2\tabcolsep}|m{\dimexpr.30\textwidth-2\tabcolsep}|m{\dimexpr.45\textwidth-2\tabcolsep}|}
\hline
\rowcolor{black!80} \multicolumn{3}{l}{\textcolor{white}{constraints}} \\

\hline
domain & \multicolumn{2}{m{\dimexpr.75\textwidth-2\tabcolsep}|}{#1} \\


\ifx\hfuzz#2\hfuzz\else
\hline
parallel domain & \multicolumn{2}{m{\dimexpr.75\textwidth-2\tabcolsep}|}{#2} \\
\fi

\ifx\hfuzz#3#4\hfuzz\else
\hline
\multirow{2}{.15\textwidth}{invocable} & \cellcolor{black!80} \textcolor{white}{default} & \cellcolor{black!80} \textcolor{white}{custom} \\
\cline{2-3}
& #3 & #4 \\
\fi

\hline
\end{tabular}
\end{center}
}