\section{Generators, copies and moves}

We will start with the prototypical generator algorithm that generates consecutive values.

\subsection{iota}

Iota is an outlier when it comes to C++20 std::ranges support. C++20 introduced the lazy view version std::views::iota, and we will only get the eager range version of the algorithm in C++23.

\cppversions{\texttt{iota}}{\CC11}{\CC20}{N/A}{\CC23}

\constraints{\texttt{forward\_range}}{}{}{}

The std::iota algorithm will generate elements by consecutively applying the prefix increment operator, starting with the initial value.

\begin{box-note}
\begin{cppcode}
std::vector<int> data(11, 0);

std::iota(data.begin(), data.end(), -5); 
// data == { -5, -4, -3, -2, -1, 0, 1, 2, 3, 4, 5 }

std::vector<int> out;
std::ranges::transform(std::views::iota(1, 10), std::views::iota(5), 
                       std::back_inserter(out), std::plus<>{});
// out == { 6, 8, 10, 12, 14, 16, 18, 20, 22 }
\end{cppcode}
\end{box-note}

Here we take advantage of the finite view constructor std::views::iota(1,10) to establish the output size (line 7), which allows us to use the infinite view std::views::iota(5) for the second parameter. Functionally, we could swap even the second view for a finite one. However, this would impose an additional (and unnecessary) boundary check.

\subsection{fill, fill\_n, generate, generate\_n}

\cppversions{\texttt{fill, generate}}{\CC98}{\CC20}{\CC17}{\CC20}

\constraints{\texttt{forward\_range}}{}{}{}

The std::fill algorithm fills the designated range with the provided value, and std::generate fills the designated range with values resulting from successive invocations of the provided invocable.

\begin{box-note}
\begin{cppcode}
std::vector<int> data(5, 0);
std::fill(data.begin(), data.end(), 11);
// data == (11, 11, 11, 11, 11)

std::ranges::generate(data, []() { return 5; });
// data == (5, 5, 5, 5, 5)

// iota-like
std::ranges::generate(data, [i = 0]() mutable { return i++; });
// data == (0, 1, 2, 3, 4)
\end{cppcode}
\end{box-note}

The iota-like behaviour (line 9) is possible because the standard guarantees successive (left-to-right) invocations of the provided invocable.

When working with ranges that do not support random access, it might be slow or impossible to provide the end of the range iterator. The standard offers variants of fill and generate where the range is specified using a begin iterator and the number of elements to cover this case.

\cppversions{\texttt{fill\_n, generate\_n}}{\CC98}{\CC20}{\CC17}{\CC20}

\constraints{\texttt{output\_iterator}}{\texttt{forward\_iterator}}{}{}

Other than the change in how a range is specified, these variants have identical behaviour to the baseline algorithms.

\begin{box-note}
\begin{cppcode}
std::vector<int> data;
std::fill_n(std::back_inserter(data), 5, 11);
// data == (11, 11, 11, 11, 11)

data.clear();
std::ranges::generate_n(std::back_inserter(data), 5, []() { return 5; });
// data == (5, 5, 5, 5, 5)
\end{cppcode}
\end{box-note}

\subsection{Iterator adapaters}

We have already used the std::back\_inserter many times across the examples in this series, so let's start with inserter adapters.

\subsubsection{back\_inserter, front\_inserter, inserter}

The standard offers three adapters that create instances of back\_insert\_iterator, front\_insert\_iterator and insert\_iterator. When assigned to, these iterators call the corresponding push\_back, push\_front and insert methods on the adapted container.

\begin{box-note}
\begin{cppcode}
std::list<int> data;
auto back = std::back_inserter(data);
auto front = std::front_inserter(data);

*back = 10;
// data == { 10 }
*front = 20;
// data == { 20, 10 }

auto mid = std::inserter(data, std::next(data.begin()));
*mid = 5;
// data == { 20, 5, 10 }
\end{cppcode}
\end{box-note}

All inserter iterators model only output iterator concept, therefore cannot be used with algorithms that require forward or bidirectional iterators.

\subsubsection{istream\_iterator, ostream\_iterator}

These two adapters provide iteration over istream and ostream, respectively.

The istream\_iterator will read a value by calling the corresponding operator>> when incremented.

\begin{box-note}
\begin{cppcode}
std::stringstream input("10 20 30 40 50");
auto in = std::istream_iterator<int>(input);

int i = *in; // i == 10
++in;
int j = *in; // j == 20
\end{cppcode}
\end{box-note}

The ostream\_iterator will write a value by calling the corresponding operator<< when assigned to.

\begin{box-note}
\begin{cppcode}
std::stringstream output;
auto out = std::ostream_iterator<int>(output, ", ");

*out = 10;
*out = 20;
// output == "10, 20,"
\end{cppcode}
\end{box-note}

The additional parameter for ostream\_iterator specifies the divider added after each element.

The istream\_iterator models an input iterator, and the ostream\_iterator models an output iterator.

\subsubsection{move\_iterator, make\_move\_iterator}

The move\_iterator adapter inherits the iterator category from the adapted iterator but will return an rvalue when dereferenced, thus enabling move semantics.

\begin{box-note}
\begin{cppcode}
std::vector<std::string> data = { "a", "b", "c" };
auto it = std::make_move_iterator(data.begin());

std::string str(*it); // move construction
++it;
str = *it; // move assignment
// data == { ?, ?, "c" }
// str == "b"
\end{cppcode}
\end{box-note}

\subsubsection{reverse\_iterator}

We can use the reverse\_iterator adapter for bidirectional iterators to obtain the iterator for the opposite direction.

\begin{box-note}
\begin{cppcode}
std::vector<int> data = { 1, 2, 3 };
auto mid = std::reverse_iterator(data.end());

int i = *mid; // i == 3
++mid;
int j = *mid; // j == 2
\end{cppcode}
\end{box-note}

Note that the reversed iterator maps to the previous element in the source direction. Therefore, when reversing begin or rbegin, the result is rend and end, respectively.

\subsubsection{counted\_iterator}

When working with non-random-access ranges, it is convenient to specify the range using a begin iterator coupled with the number of elements. The counted\_iterator brings this feature to C++20 ranges without the need for specialised algorithm variants like std::fill\_n.

\begin{box-note}
\begin{cppcode}
std::list<int> data = { 1, 2, 3, 4, 5 };

std::ranges::fill(std::counted_iterator(data.begin(), 3), std::default_sentinel, 9);
// data == { 9, 9, 9, 4, 5 }
\end{cppcode}
\end{box-note}

The counted\_iterator keeps track of the number of elements, and when the counter reaches zero, the counted\_iterator will compare equal to std::default\_sentinel, signifying the end of the range.

\subsection{copy, move, copy\_backwards, move\_backwards}

When copying/moving a range, we need to avoid overwriting the yet to be copied/moved elements from the source range. The standard offers two directions of copying to prevent this issue. We will start with the forward version:

\cppversions{\texttt{copy}}{\CC98}{\CC20}{\CC17}{\CC20}
\cppversions{\texttt{move}}{\CC11}{\CC20}{\CC17}{\CC20}

\constraints{\texttt{input\_range -> output\_iterator}}{\texttt{forward\_range -> forward\_iterator}}{}{}

The output iterator is not permitted to be within [first, last) (of the input range). Consequently, only the tail of the output range can overlap with the input range.

\begin{box-note}
\begin{cppcode}
std::vector<std::string> data{ "a", "b", "c", "d", "e", "f"};

std::copy(data.begin(), data.begin()+3, data.begin()+3);
// data = { "a", "b", "c", "a", "b", "c" }

// Overlapping case:
std::copy(std::next(data.begin()), data.end(), data.begin());
// data = { "b", "c", "a", "b", "c", "c" }
\end{cppcode}
\end{box-note}

Move operates identically, except it casts each element to an rvalue before the assignment, turning copies into moves.

\begin{box-note}
\begin{cppcode}
std::vector<std::string> data{ "a", "b", "c", "d", "e", "f"};

std::move(data.begin(), data.begin()+3, data.begin()+3);
// data = { ?, ?, ?, "a", "b", "c" }
\end{cppcode}
\end{box-note}

Significantly, whether std::move will move depends on the underlying element type. If the underlying type is copy-only, std::move will behave identically to std::copy.

\begin{box-note}
\begin{cppcode}
struct CopyOnly {
    CopyOnly() = default;
    CopyOnly(const CopyOnly&) = default;
    CopyOnly& operator=(const CopyOnly&) { 
      std::cout << "Copy assignment.\n";
      return *this;
    };
};

std::vector<CopyOnly> test(6);

std::move(test.begin(), test.begin()+3, test.begin()+3);
// 3x Copy assignment
\end{cppcode}
\end{box-note}

We can also emulate std::move by using the previously mentioned move\_iterator adapter.

\begin{box-note}
\begin{cppcode}
std::vector<std::string> data{ "a", "b", "c", "d", "e", "f"};

std::copy(std::make_move_iterator(data.begin()), std::make_move_iterator(data.begin()+3), 
          data.begin()+3);
// data = { ?, ?, ?, "a", "b", "c" }
\end{cppcode}
\end{box-note}

When we need to copy to an overlapping range that isn’t suitable for forward copy, we can use the backward variant that copies in the opposite direction.

\cppversions{\texttt{copy\_backward}}{\CC98}{\CC20}{N/A}{\CC20}
\cppversions{\texttt{move\_backward}}{\CC11}{\CC20}{N/A}{\CC20}

\constraints{\texttt{bidirectional\_range -> bidirectional\_iterator}}{}{}{}

The output iterator cannot be within (first, last] and will be treated as the end iterator for the destination range, meaning that the algorithm will write the first value to std::prev(end).

\begin{box-note}
\begin{cppcode}
std::vector<std::string> data{ "a", "b", "c", "d", "e", "f"};

std::vector<std::string> out(9, "");
std::copy_backward(data.begin(), data.end(), out.end());
// "", "", "", "a", "b", "c", "d", "e", "f"

std::copy_backward(data.begin(), std::prev(data.end()), data.end());
// data = { "a", "a", "b", "c", "d", "e" }
\end{cppcode}
\end{box-note}

Similar to fill\_n and generate\_n, the standard also provides a copy\_n variant suitable for non-random-access ranges.

\cppversions{\texttt{copy\_n}}{\CC11}{\CC20}{\CC17}{\CC20}

\constraints{\texttt{input\_iterator -> output\_iterator}}{\texttt{forward\_iterator -> forward\_iterator}}{}{}

The algorithm cannot check whether the requested count is valid and does not go out of bounds, so this burden is on the caller.

\begin{box-note}
\begin{cppcode}
std::list<int> data{1, 2, 3, 4, 5, 6, 7, 8, 9};
std::vector<int> out;

std::copy_n(data.begin(), 5, std::back_inserter(out));
// out == { 1, 2, 3, 4, 5 }
\end{cppcode}
\end{box-note}

\subsection{copy\_if, remove\_copy, remove\_copy\_if}

We can use copy\_if or remove\_copy\_if when we need to select only some of the elements to be copied.

\cppversions{\texttt{copy\_if}}{\CC11}{\CC20}{\CC17}{\CC20}
\cppversions{\texttt{remove\_copy, remove\_copy\_if}}{\CC98}{\CC20}{\CC17}{\CC20}

\constraints{\texttt{input\_range -> output\_range}}{\texttt{forward\_range -> forward\_iterator}}{N/A}{\texttt{unary\_predicate}}

These two algorithms provide the same logic, with opposite semantics: copy\_if will copy elements for which the predicate returns true, and remove\_copy\_if will copy elements for which the predicate returns false. Finally, remove\_copy copies elements that do not match the provided value.

\begin{box-note}
\begin{cppcode}
std::vector<int> data{ 1, 2, 3, 4, 5, 6, 7, 8, 9};
std::vector<int> even, odd, no_five;

auto is_even = [](int v) { return v % 2 == 0; };

std::ranges::copy_if(data, std::back_inserter(even), is_even);
// even == { 2, 4, 6, 8 }

std::ranges::remove_copy_if(data, std::back_inserter(odd), is_even);
// odd == { 1, 3, 5, 7, 9 }

std::ranges::remove_copy(data, std::back_inserter(no_five), 5);
// no_five == { 1, 2, 3, 4, 6, 7, 8, 9 }
\end{cppcode}
\end{box-note}

\subsection{sample}

A different twist on a selective copy is the sample algorithm.

\cppversions{\texttt{sample}}{\CC17}{N/A}{N/A}{\CC20}

\constraints{\texttt{forward\_range -> output\_iterator}}{}{}{}
\constraints{\texttt{input\_range -> random\_access\_iterator}}{}{}{}

The sample algorithm will copy a random selection of N elements from the source range to the destination range utilising the provided random number generator.

\begin{box-note}
\begin{cppcode}
std::vector<int> data{1, 2, 3, 4, 5, 6, 7, 8, 9};
std::vector<int> out;

std::sample(data.begin(), data.end(), std::back_inserter(out),
            5, std::mt19937{std::random_device{}()});
// e.g. out == {2, 3, 4, 5, 9} (guaranteed ascending, because source range is ascending)
\end{cppcode}
\end{box-note}

The two domains of this algorithm are due to the stable nature of the sampling, maintaining the order of elements from the source range.

This feature requires either the input range to be at least a forward range or the destination range needs to be a random-access range.

\subsection{replace\_copy, replace\_copy\_if}

We discussed the replace algorithm in a previous article in the series. The copy variants work identical, replacing values matching a predicate or a provided value. However, the result is output to a destination range instead of the replacement applied in place.

\cppversions{\texttt{replace\_copy, replace\_copy\_if}}{\CC98}{\CC20}{\CC17}{\CC20}

\constraints{\texttt{input\_range -> output\_iterator}}{\texttt{forward\_range -> forward\_iterator}}{N/A}{\texttt{unary\_predicate}}

The replace\_copy\_if variant replaces elements for which the predicate returns true. The replace\_copy replaces elements that match the provided value.

\begin{box-note}
\begin{cppcode}
std::vector<int> data{1, 2, 3, 4, 5, 6, 7, 8, 9};
std::vector<int> out, odd;

std::ranges::replace_copy(data, std::back_inserter(out), 5, 10);
// out == { 1, 2, 3, 4, 10, 6, 7, 8, 9 }

auto even = [](int v) { return v % 2 == 0; };
std::ranges::replace_copy_if(data, std::back_inserter(odd), even, -1);
// odd == { 1, -1, 3, -1, 5, -1, 7, -1, 9 }
\end{cppcode}
\end{box-note}

\subsection{reverse\_copy, rotate\_copy}

The last two algorithms from the copiers category change the order of copied elements.

\cppversions{\texttt{reverse\_copy}}{\CC98}{\CC20}{\CC17}{\CC20}

\constraints{\texttt{bidirectional\_range -> output\_iterator}}{\texttt{bidirectional\_range -> forward\_iterator}}{}{}

Not to be confused with copy\_backwards, which maintains the original order of elements, reverse\_copy will reverse the order of elements copied.

\begin{box-note}
\begin{cppcode}
std::vector<int> data{1, 2, 3, 4, 5, 6, 7, 8, 9};
std::vector<int> out;

std::ranges::reverse_copy(data, std::back_inserter(out));
// out == { 9, 8, 7, 6, 5, 4, 3, 2, 1 }
\end{cppcode}
\end{box-note}

\cppversions{\texttt{rotate\_copy}}{\CC98}{\CC20}{\CC17}{\CC20}

\constraints{\texttt{(forward\_range, forward\_iterator) -> output\_iterator}}{\texttt{(forward\_range, forward\_iterator) -> forward\_iterator}}{}{}

Following the logic of the rotate algorithm, rotate\_copy will first copy the elements starting from the designated middle element and then follow with the remaining elements from the beginning of the range.

\begin{box-note}
\begin{cppcode}
std::vector<int> data{1, 2, 3, 4, 5, 6, 7, 8, 9};
std::vector<int> out;

std::ranges::rotate_copy(data, data.begin()+4, std::back_inserter(out));
// out == { 5, 6, 7, 8, 9, 1, 2, 3, 4 }
\end{cppcode}
\end{box-note}
