\section{Transformation algorithms}

In this section, we will discuss algorithms that transform ranges by changing the values of elements and removing and re-ordering elements.

\subsection{\texorpdfstring{\cpp{std::transform}}{\texttt{std::transform}}}
\index{\cpp{std::transform}}

The most straightforward transformation possible is to apply a transformation function to each element. The \cpp{std::transform} algorithm provides this functionality in unary and binary variants (input from one or two ranges).

\cppversions{\texttt{transform}}{\CC98}{\CC20}{\CC17}{\CC20}

% Mention lazy version

\constraints{\texttt{input\_range -> output\_iterator}\newline\texttt{(input\_range, input\_iterator) -> output\_iterator}}{\texttt{forward\_range -> forward\_iterator}\newline\texttt{(forward\_range, forward\_iterator) -> forward\_iterator}}{N/A}{\texttt{unary\_functor}\newline\texttt{binary\_functor}}

\begin{box-note}
\footnotesize Example of unary and binary version of \cpp{std::transform}. Note that the output iterator can be one of the input ranges' begin iterator (line 4 and 12).
\tcblower
\cppfile{code_examples/algorithms/transform_code.h}
\end{box-note}

Note that \cpp{std::transform} does not guarantee strict left-to-right evaluation. If that is required, use \cpp{std::for_each} instead.

\subsection{\texorpdfstring{\cpp{std::adjacent_difference}}{\texttt{std::adjacent\_difference}}}
\index{\cpp{std::adjacent_difference}}

The \cpp{std::adjacent_difference} is a numerical algorithm, which might make it seem out of place in this section; however, it is essentially a binary transformation algorithm that operates on adjacent elements of the single source range.

Unlike \cpp{std::transform}, \cpp{std::adjacent_difference} guarantees left-to-right application. Furthermore, because it operates on input ranges, the algorithm internally stores a copy of the last read value for use as the left operand in the next step.

\cppversions{\texttt{adjacent\_difference}}{\CC98}{\CC20}{\CC17}{N/A}
\constraints{\texttt{input\_range -> output\_iterator}}{\texttt{forward\_range -> forward\_iterator}}{\texttt{operator -}}{\texttt{binary\_functor}}

\begin{box-note}
\footnotesize Example of the default version of \cpp{std::adjacent_difference}, which will calculate the difference of adjacent elements, with the first element copied.
\tcblower
\cppfile{code_examples/algorithms/adjacent_difference_code.h}
\end{box-note}

\begin{box-note}
\footnotesize Example of more inventive use of \cpp{std::adjacent_difference} to generate the Fibonacci sequence.
\tcblower
\cppfile{code_examples/algorithms/adjacent_difference_extra_code.h}
\end{box-note}

Note that the \cpp{std::next(data.begin())} for the output range is critical here. The \cpp{std::adjacent_difference} algorithm will read each element only once and remembers the previously read value for the left argument. The \cpp{std::next} ensures that we generate one element ahead of either argument.

\subsection{\texorpdfstring{\cpp{std::remove}, \cpp{std::remove_if}}{\texttt{std::remove}, \texttt{std::remove\_if}}}
\index{\cpp{std::remove}}
\index{\cpp{std::remove_if}}

The \cpp{std::remove} and \cpp{std::remove_if} algorithms "remove" elements that match the given value or for which the given predicate evaluates to true.

Because the algorithms cannot resize the underlying range, the removal is achieved by moving the other elements in the range. The algorithms then return an iterator beyond the last not removed element, i.e. the new end iterator.

\cppversions{\texttt{remove, remove\_if}}{\CC98}{\CC20}{\CC17}{\CC20}
\constraints{\texttt{forward\_range}}{\texttt{forward\_range}}{N/A}{\texttt{unary\_predicate}}

\begin{box-note}
\footnotesize Example of using \cpp{std::remove} and \cpp{std::remove_if}.
\tcblower
\cppfile{code_examples/algorithms/remove_code.h}
\end{box-note}

\subsection{\texorpdfstring{\cpp{std::replace}, \cpp{std::replace_if}}{\texttt{std::replace}, \texttt{std::replace\_if}}}
\index{\cpp{std::replace}}
\index{\cpp{std::replace_if}}

The \cpp{std::replace} and \cpp{std::replace_if} algorithms replace elements that match the given value or for which the given predicate evaluates to true.

\cppversions{\texttt{replace, replace\_if}}{\CC98}{\CC20}{\CC17}{\CC20}
\constraints{\texttt{forward\_range}}{\texttt{forward\_range}}{N/A}{\texttt{unary\_predicate}}

\begin{box-note}
\footnotesize Example of using \cpp{std::replace} and \cpp{std::replace_if}.
\tcblower
\cppfile{code_examples/algorithms/replace_code.h}
\end{box-note}

\subsection{\texorpdfstring{\cpp{std::reverse}}{\texttt{std::reverse}}}
\index{\cpp{std::reverse}}

The \cpp{std::reverse} algorithm will reverse the order of elements in the range by applying \cpp{std::swap} to pairs of elements.

\cppversions{\texttt{reverse}}{\CC98}{\CC20}{\CC17}{\CC20}
\constraints{\texttt{bidirectional\_range}}{\texttt{bidirectional\_range}}{}{}

Note that using \cpp{std::reverse} is only a reasonable solution if the range must be mutated because bidirectional ranges already support reverse iteration.

\begin{box-note}
\footnotesize Example of using \cpp{std::reverse} and reverse iteration, provided by bidirectional ranges.
\tcblower
\cppfile{code_examples/algorithms/reverse_code.h}
\end{box-note}

C-style arrays and C-style strings can be adapted using \cpp{std::span} and \cpp{std::string_view} to allow reverse iteration.

\begin{box-note}
\footnotesize Example of using \cpp{std::span} and \cpp{std::string_view} to addapt C-style constructs for reverse iteration.
\tcblower
\cppfile{code_examples/extras/span_stringview_code.h}
\end{box-note}

\subsection{\texorpdfstring{\cpp{std::rotate}}{\texttt{std::rotate}}}
\index{\cpp{std::rotate}}

The \cpp{std::rotate} algorithm rearranges elements in the range from \cpp{[first, middle),[middle, last)} to \cpp{[middle, last),[first, middle)}.

\cppversions{\texttt{rotate}}{\CC11}{\CC20}{\CC17}{\CC20}
\constraints{\texttt{(forward\_range, forward\_iterator)}}{\texttt{(forward\_range, forward\_iterator)}}{}{}

\begin{box-note}
\footnotesize Example of using \cpp{std::rotate}.
\tcblower
\cppfile{code_examples/algorithms/rotate_code.h}
\end{box-note}

\subsection{\texorpdfstring{\cpp{std::shuffle}}{\texttt{std::shuffle}}}
\index{\cpp{std::shuffle}}

The \cpp{std::shuffle} algorithm is a successor of the now-defunct \cpp{std::random_shuffle} algorithm (deprecated in \CC14, removed in \CC17) and relies on new random facilities added in \CC11.

\cppversions{\texttt{shuffle}}{\CC11}{N/A}{N/A}{\CC20}
\constraints{\texttt{random\_access\_range}}{}{}{}

\begin{box-note}
\footnotesize Example of using \cpp{std::shuffle} to shuffle a deck of cards.
\tcblower
\cppfile{code_examples/algorithms/shuffle_code.h}
\end{box-note}

\subsection{\texorpdfstring{\cpp{std::next_permutation}, \cpp{std::prev_permutation}}{\texttt{std::next\_permutation}, \texttt{std::prev\_permutation}}}
\index{\cpp{std::next_permutation}}
\index{\cpp{std::prev_permutation}}

The \cpp{std::next_permutation} and \cpp{std::prev_permutation} algorithms will rearrange the element so that they are in their next or previous (by lexicographical comparison) permutation. When there is no such ordering, both algorithms will wrap around, but will also return \cpp{false} to notify the caller.

\cppversions{\texttt{next\_permutation}}{\CC98}{\CC20}{N/A}{\CC20}
\cppversions{\texttt{prev\_permutation}}{\CC98}{\CC20}{N/A}{\CC20}
\constraints{\texttt{bidirectional\_range}}{}{\texttt{operator <}}{\texttt{strict\_weak\_ordering}}

\begin{box-note}
\footnotesize Example of using \cpp{std::next_permutation} to iterate over all permutations of three unique elements.
\tcblower
\cppfile{code_examples/algorithms/next_permutation_code.h}
\end{box-note}

\subsection{\texorpdfstring{\cpp{std::is_permutation}}{\texttt{std::is\_permutation}}}
\index{\cpp{std::is_permutation}}

The \cpp{std::is_permutation} algorithm is an excellent tool for checking whether two ranges have the same content but not necessarily the same order of elements.

\cppversions{\texttt{is\_permutation}}{\CC11}{\CC20}{\CC17}{\CC20}
\constraints{\texttt{(forward\_range, forward\_range)}}{\texttt{(forward\_range, forward\_range)}}{\texttt{operator==}}{\texttt{binary\_predicate}}

\begin{box-note}
\footnotesize Example of using \cpp{std::is_permutation} to validate a simple sort implementation.
\tcblower
\cppfile{code_examples/algorithms/is_permutation_code.h}
\end{box-note}
