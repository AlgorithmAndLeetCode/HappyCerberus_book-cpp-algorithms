\section{Divide and conquer}

Divide and conquer algorithms offer a great mix of performance and functionality.

While we can utilize hash-based containers to lookup any specific element in $O(1)$ amortized time, this approach has two drawbacks. Firstly, we can only look up a specific element; if that element is not present in the container, we get a simple lookup failure. Secondly, our type must be hashable, and the hash function must be reasonably fast.

Divide and conquer algorithms allow the lookup of bounds based on strict weak ordering and work even when the container’s specific value is not present. Additionally, since we are working with a sorted container, we can easily access neighbouring values once we have determined a boundary.

\subsection{\texorpdfstring{\cpp{std::lower_bound}, \cpp{std::upper_bound}}{\texttt{std::lower\_bound}, \texttt{std::upper\_bound}}}

The \cpp{std::lower_bound} and \cpp{std::upper_bound} algorithms offer boundary search with logarithmic complexity (for random access ranges.

\cppversions{\texttt{lower\_bound}}{\CC98}{\CC20}{N/A}{\CC20}
\cppversions{\texttt{upper\_bound}}{\CC98}{\CC20}{N/A}{\CC20}
\constraints{\texttt{forward\_range}}{}{\texttt{operator<}}{\texttt{strict\_weak\_ordering}}

\noindent The two algorithms differ in which bound they return:

\begin{itemize}
    \item the \cpp{std::lower_bound} returns the first element for which \texttt{elem < value} returns false (i.e. first element for which \texttt{elem >= value})
    \item the \cpp{std::upper_bound} returns the first element for which \texttt{value < elem}
    \item if no such element exists, both algorithms return the end iterator
\end{itemize}

\begin{box-note}
\footnotesize Example of using \cpp{std::lower_bound} and \cpp{std::upper_bound} to divide a sorted range into three parts: lower than the bottom threshold, between the bottom and upper threshold and higher than the upper threshold.
\tcblower
\cppfile{code_examples/algorithms/lower_bound_code.h}
\end{box-note}

While the algorithms will operate on any \cpp{forward_range}, the logarithmic divide and conquer behaviour is only available for \cpp{random_access_range}. Data structures like \cpp{std::set}, \cpp{std::multiset}, \cpp{std::map} and \cpp{std::multimap} offer their $O(logn)$ implementations of lower and upper bound as methods.

\begin{box-note}
\footnotesize Example of using \cpp{lower_bound} and \cpp{upper_bound} methods on a \cpp{std::multiset}.
\tcblower
\cppfile{code_examples/algorithms/lower_bound_set_code.h}
\end{box-note}

\subsection{\texorpdfstring{\cpp{std::equal_range}}{\texttt{std::equal\_range}}}

The \cpp{std::equal_range} algorithm returns both lower and upper bounds for the given value.

\cppversions{\texttt{equal\_range}}{\CC98}{\CC20}{N/A}{\CC20}
\constraints{\texttt{forward\_range}}{}{\texttt{operator<}}{\texttt{strict\_weak\_ordering}}

Because the lower bound returns the first element for which \texttt{elem >= value} and the upper bound returns the first element for which \texttt{value < elem}, the result is a range \texttt{[lb, ub)} of elements equal to the value.

\begin{box-note}
\footnotesize Example of using \cpp{std::equal_range}.
\tcblower
\cppfile{code_examples/algorithms/equal_range_code.h}
\end{box-note}

\subsection{\texorpdfstring{\cpp{std::partition_point}}{\texttt{std::partition\_point}}}

Despite the naming, \cpp{std:partition_point} works very similarly to \cpp{std::upper_bound}, however instead of searching for a particular value, it searches using a predicate.

\cppversions{\texttt{partition\_point}}{\CC11}{\CC20}{N/A}{\CC20}
\constraints{\texttt{forward\_range}}{}{N/A}{\texttt{unary\_predicate}}

\cpp{std::partition_point} will return the first element that does not satisfy the provided predicate. This algorithm only requires the range to be partitioned (with respect to the predicate).

\begin{box-note}
\footnotesize Example of using \cpp{std::partition_point}.
\tcblower
\cppfile{code_examples/algorithms/partition_point_code.h}
\end{box-note}

\subsection{\texorpdfstring{\cpp{std::binary_search}}{\texttt{std::binary\_search}}}

The \cpp{std::binary_search} provides a presence check, returning a boolean indicating whether the requested value is present in the sorted range or not.

\cppversions{\texttt{binary\_search}}{\CC98}{\CC20}{N/A}{\CC20}
\constraints{\texttt{forward\_range}}{}{\texttt{operator<}}{\texttt{strict\_weak\_ordering}}

Using \cpp{std::binary_search} is equivalent to calling \cpp{std::equal_range} and checking whether the returned is non-empty; however, \cpp{std::binary_search} offers a single lookup performance, where \cpp{std::equal_range} does two lookups to determine the lower and upper bounds.

\begin{box-note}
\footnotesize Example of using \cpp{std::binary_search} with an equivalent check using \cpp{std::equal_range}.
\tcblower
\cppfile{code_examples/algorithms/binary_search_code.h}
\end{box-note}


\subsection{\texorpdfstring{\cpp{bsearch} - C standard library}{\texttt{bsearch} - C standard library}}

From the C standard library, C++ inherits bsearch. This algorithm returns one of the elements equal to the provided key, or nullptr if none such element is found.

\begin{box-note}
\begin{cppcode}
int data[] = {-2, -1, 0, 1, 2};
int size = sizeof data / sizeof(int);

auto cmp = [](const void* left, const void* right){
    int vl = *(const int*)left;
    int vr = *(const int*)right;

    if (vl < vr) return -1;
    if (vl > vr) return 1;
    return 0;
};

int value = 1;
void* el1 = bsearch(&value, data, size, sizeof(int), cmp);
assert(*static_cast<int*>(el1) == 1);

value = 3;
void *el2 = bsearch(&value, data, size, sizeof(int), cmp); 
assert(el2 == nullptr);
\end{cppcode}
\end{box-note}

As with qsort, there is effectively no reason to use bsearch in C++ code.

Depending on the specific use case, one of the previously mentioned algorithms should be a suitable replacement.

\begin{box-note}
\begin{cppcode}
int data[] = {-2, -1, 0, 1, 2};
int size = sizeof data / sizeof(int);

int value = 1;
bool exist = std::binary_search(&data[0], &data[size], value);

auto candidate = std::lower_bound(&data[0], &data[size], value);
if (candidate != &data[size] && *candidate == value) {
    // process element
}

auto [lb, ub] = std::equal_range(&data[0], &data[size], value);
if (lb != ub) {
    // process equal elements
}
\end{cppcode}
\end{box-note}

\subsection{includes}

The first linear algorithm we will talk about is std::includes. This algorithm determines whether one range is a sub-range of another. Since we are working on sorted ranges, std::includes runs in linear time.

\cppversions{\texttt{includes}}{\CC98}{\CC20}{\CC17}{\CC20}

\constraints{\texttt{(input\_range, input\_range)}}{\texttt{forward\_range, forward\_range)}}{\texttt{operator<}}{\texttt{strict\_weak\_ordering}}

Here we check whether the input text contains all the lowercase English letters.

\begin{box-note}
\begin{cppcode}
std::vector<char> letters('z'-'a'+1,'\0');
std::iota(letters.begin(), letters.end(), 'a');

std::string input = "the quick brown fox jumps over the lazy dog";
std::ranges::sort(input);

assert(std::ranges::includes(input, letters));
\end{cppcode}
\end{box-note}

First, we generate the list of English letters programmatically using another algorithm std::iota (line 2). The iota algorithm generates consecutively increasing values to fill the given range. Because of this, we preallocate the vector to 26 elements (line 1).

\subsection{merge, inplace\_merge}

Another operation that is feasible in linear time is merging two sorted ranges.

\cppversions{\texttt{merge}}{\CC98}{\CC20}{\CC17}{\CC20}

\constraints{\texttt{(input\_range, input\_range)}}{\texttt{(forward\_range, forward\_range)}}{\texttt{operator<}}{\texttt{strict\_weak\_ordering}}

The result of the merge operation is stored using the provided output iterator. Note that the output range is not permitted to overlap with either of the input ranges.

The merge operation is stable. Equal elements from the first range will be ordered before equal elements from the second range.

\begin{box-note}
\begin{cppcode}
struct LabeledValue {
    int value;
    std::string label;
};

std::vector<LabeledValue> data1{{1, "first"}, {2, "first"}, {3, "first"}};
std::vector<LabeledValue> data2{{0, "second"}, {2, "second"}, {4, "second"}};

std::vector<LabeledValue> result;
std::ranges::merge(data1, data2, std::back_inserter(result),
  [](const auto& left, const auto& right) { return left.value < right.value; });
// result == {0, second}, {1, first}, {2, first}, {2, second}, {3, first}, {4, second}
\end{cppcode}
\end{box-note}

The parallel version requires the output to be a forward range (represented by a forward\_iterator). Therefore, we cannot use wrappers like std::back\_inserter and must preallocate the output range to sufficient capacity.

\begin{box-note}
\begin{cppcode}
std::vector<int> data1{1, 2, 3, 4, 5, 6};
std::vector<int> data2{3, 4, 5, 6, 7, 8};

std::vector<int> out(data1.size()+data2.size(), 0);
std::merge(std::execution::par_unseq,
    data1.begin(), data1.end(),
    data2.begin(), data2.end(),
    out.begin());
\end{cppcode}
\end{box-note}

\cppversions{\texttt{inplace\_merge}}{\CC98}{N/A}{\CC17}{\CC20}

\constraints{\texttt{(bidirectional\_range, bidirectional\_iterator)}}{}{\texttt{operator<}}{\texttt{inplace\_merge}}

Because merge forbids the input and output ranges from overlapping, we have an alternative inplace\_merge that serves this use case.

\begin{box-note}
\begin{cppcode}
std::vector<int> range{1, 3, 5, 2, 4, 6};
std::inplace_merge(range.begin(), range.begin()+3, range.end());
// range == { 1, 2, 3, 4, 5, 6 }
\end{cppcode}
\end{box-note}

\subsection{unique, unique\_copy}

The std::unique algorithm removes consecutive duplicate values. The typical use case is in conjunction with a sorted range. However, this is not a requirement of std::unique.

\cppversions{\texttt{unique}}{\CC98}{\CC20}{\CC17}{\CC20}

\constraints{\texttt{forward\_range}}{}{\texttt{operator==}}{\texttt{binary\_predicate}}

Because unique works in-place and cannot resize the underlying range, it leaves the end of the range with unspecified values and returns an iterator to the beginning of this sub-range (or the sub-range in case of the range version).

\begin{box-note}
\begin{cppcode}
std::vector<int> data{1, 1, 2, 2, 3, 4, 5, 6, 6, 6};
auto it = std::unique(data.begin(), data.end());
// Range version: auto it = std::ranges::unique(data).begin();

// data == {1, 2, 3, 4, 5, 6, unspec, unspec, unspec, unspec}
data.resize(std::distance(data.begin(), it));
// data == {1, 2, 3, 4, 5, 6}
\end{cppcode}
\end{box-note}

\cppversions{\texttt{unique\_copy}}{\CC98}{\CC20}{\CC17}{\CC20}

\constraints{\texttt{input\_range -> output\_iterator}}{\texttt{forward\_range -> forward\_iterator}}{\texttt{operator==}}{\texttt{binary\_predicate}}

The copy version of unique instead outputs the unique values to an output range represented by an iterator.

\begin{box-note}
\begin{cppcode}
std::vector<int> data{1, 1, 2, 2, 3, 4, 5, 6, 6, 6};
std::vector<int> out;
std::ranges::unique_copy(data, std::back_inserter(out));
// out == {1, 2, 3, 4, 5, 6}
\end{cppcode}
\end{box-note}

\subsection{set\_difference, set\_symmetric\_difference, set\_union, set\_intersection}

The last group of algorithms that require sorted ranges are set operations.

\cppversions{\texttt{set\_difference}}{\CC98}{\CC20}{\CC17}{\CC20}
\cppversions{\texttt{set\_symmetric\_difference}}{\CC98}{\CC20}{\CC17}{\CC20}
\cppversions{\texttt{set\_union}}{\CC98}{\CC20}{\CC17}{\CC20}
\cppversions{\texttt{set\_intersection}}{\CC98}{\CC20}{\CC17}{\CC20}

All the set operations work in the same way, processing two input ranges and copying results into the result range, with the following semantics:

\begin{description}
   \item[\texttt{set\_difference}] elements that are present in the first range, but not the second
   \item[\texttt{set\_symmetric\_difference}] elements that are present in only one of the ranges, but not both
   \item[\texttt{set\_union}] elements that are present in either of the ranges
   \item[\texttt{set\_intersection}] elements that are present in both ranges
\end{description}

We also need to talk about the behaviour for equal elements. If we have m such elements in the first range and n such elements in the second range, the resulting range will contain:


\begin{description}
   \item[\texttt{set\_difference}] $max(m-n,0)$ elements
   \item[\texttt{set\_symmetric\_difference}] $abs(m-n)$, that is: if $m>n$, $m-n$ elements will be copied from the first range; otherwise, $n-m$ elements will be copied from the second range
   \item[\texttt{set\_union}] $max(m,n)$, first $m$ elements will be copied from the first range, followed by $max(n-m,0)$ elements from the second range
   \item[\texttt{set\_intersection}] $min(m,n)$, elements will be copied from the first range
\end{description}

\begin{box-note}
\begin{cppcode}
std::vector<int> data1{1, 2, 5};
std::vector<int> data2{2, 4, 6};

std::vector<int> difference;
std::ranges::set_difference(data1, data2, std::back_inserter(difference));
// difference == {1, 5}

std::vector<int> symmetric;
std::ranges::set_symmetric_difference(data1, data2, std::back_inserter(symmetric));
// symmetric == {1, 4, 5, 6}

std::vector<int> set_union;
std::ranges::set_union(data1, data2, std::back_inserter(set_union));
// set_union == {1, 2, 4, 5, 6}

std::vector<int> intersection;
std::ranges::set_intersection(data1, data2, std::back_inserter(intersection));
// intersection == {2}
\end{cppcode}
\end{box-note}