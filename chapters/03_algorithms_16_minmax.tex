\section{Min-Max algorithms}

This group contains algorithms that operate with minimums and maximums. However, two theoretical topics are relevant here, and we need to talk about them first: \cpp{std::initializer_list} and \cpp{const_cast}.

For functions that accept an \cpp{std::initializer_list}, it is worth keeping in mind that \cpp{std::initializer_list} is constructed by copy; its internal array of elements is copy constructed from the listed elements. Therefore we need to be careful when using \cpp{std::initializer_list} outside of compile-time contexts.

\begin{box-note}
\footnotesize Example demonstrating case when utilizing \cpp{std::initializer_list} leads to excessive copies.
\tcblower
\cppfile{code_examples/theory/initializer_list_code.h}
\end{box-note}

In this example, using \cpp{std::initializer_list} leads to six copies (which we count using the static data member \cpp{X::copy_cnt}). Five copies result from passing in the variables \texttt{a} to \texttt{e} into the \cpp{std::initializer_list}, and one is the result of the return from \cpp{std::max}.

In rare cases, we can force constness on a mutable entity. If the constness is undesirable, using \cpp{const_cast} to cast away the const is an option.

\begin{box-note}
\footnotesize Example demonstrating the valid and invalid uses for \cpp{const_cast}.
\tcblower
\cppfile{code_examples/theory/const_cast_code.h}
\end{box-note}

Please remember that when using casts like \cpp{const_cast}, you effectively override the compiler's judgment. Therefore it is entirely up to you to ensure that the given cast is valid.

\subsection{\texorpdfstring{\cpp{std::min}, \cpp{std::max}, \cpp{std::minmax}}{\texttt{std::min}, \texttt{std::max}, \texttt{std::minmax}}}

The basic versions of \cpp{std::min}, \cpp{std::max} and \cpp{std::minmax} operate on two elements, accepting their arguments by const-reference and returning by const-reference. Unfortunately, as mentioned earlier, this creates a constness problem, and we also must be careful to capture the result by value when passing in temporary objects.

\cppversions{\texttt{min, max}}{\CC98}{\CC14}{N/A}{\CC20}
\cppversions{\texttt{minmax}}{\CC11}{\CC14}{N/A}{\CC20}

\begin{box-note}
\footnotesize Example demonstrating use of \cpp{std::min} and \cpp{std::max}.
\tcblower
\cppfile{code_examples/algorithms/min_max_code.h}
\end{box-note}

Capturing the result by value gets a bit more complicated with \cpp{std::minmax}, which returns a \cpp{std::pair} of const references. To avoid dangling references to expired prvalues we must explicitly name the result type. Unfortunately, there is no way to work around this problem when using \cpp{auto} or structured binding.

\begin{box-note}
\footnotesize Example demonstrating use of \cpp{std::minmax}.
\tcblower
\cppfile{code_examples/algorithms/minmax_code.h}
\end{box-note}

The C++14 and C++20 standards introduced additional variants of the min-max algorithms that return by value. The change to return by value resolves the dangling reference issue while simultaneously introducing the potential for excessive copies.

\constraints{\CC14: \texttt{initializer\_list}\newline \CC20 range version: \texttt{input\_range}}{}{\texttt{operator<}}{\texttt{strict\_weak\_ordering}}

\begin{box-note}
\footnotesize Example of \cpp{std::initializer_list} and range variants of \cpp{std::min}, \cpp{std::max} and \cpp{std::minmax}.
\tcblower
\cppfile{code_examples/algorithms/minmax_extra_code.h}
\end{box-note}

\subsection{\texorpdfstring{\cpp{std::clamp}}{\texttt{std::clamp}}}

The \cpp{std::clamp} algorithm takes three arguments, the value, the minimum and the maximum bound and will clamp the value between the provided minimum and maximum:

\cppversions{\texttt{clamp}}{\CC17}{\CC17}{N/A}{\CC20}

\begin{itemize}
    \item if $value < minimum$, \cpp{std::clamp} returns the minimum
    \item if $maximum < value$, \cpp{std::clamp} returns the maximum
    \item otherwise, \cpp{std::clamp} returns the value
\end{itemize}

Because the algorithm accepts its arguments by const reference and returns a const reference, it shares the same issues as the min-max algorithms regarding const correctness and dangling references.

\begin{box-note}
\footnotesize Examples of using \cpp{std::clamp} with prvalues and with lvalues and \cpp{const_cast} to get mutable access to the original variable.
\tcblower
\cppfile{code_examples/algorithms/clamp_code.h}
\end{box-note}


\subsection{\texorpdfstring{\cpp{std::min_element}, \cpp{std::max_element}, \cpp{std::minmax_element}}{\texttt{std::min\_element}, \texttt{std::max\_element}, \texttt{std::minmax\_element}}}

The element versions of min-max algorithms operate on ranges and, instead of returning by const-reference or value, return an iterator to the minimum or maximum elements.

\cppversions{\texttt{min\_element}}{\CC98}{\CC20}{\CC17}{\CC20}
\cppversions{\texttt{max\_element}}{\CC98}{\CC20}{\CC17}{\CC20}
\cppversions{\texttt{minmax\_element}}{\CC11}{\CC20}{\CC17}{\CC20}

\constraints{\texttt{forward\_range}}{\texttt{forward\_range}}{\texttt{operator<}}{\texttt{strict\_weak\_ordering}}

\begin{box-note}
\footnotesize Example of using element versions of min-max algorithms.
\tcblower
\cppfile{code_examples/algorithms/min_element_code.h}
\end{box-note}

You might be wondering whether we can cause the same dangling reference (iterator) issue with the element versions of min-max algorithms. Fortunately, one very nice feature of C++20 ranges is the protection against precisely this problem.

\begin{box-note}
\footnotesize Example demonstrating protection from dangling iterators.
\tcblower
\cppfile{code_examples/algorithms/min_element_dangling_code.h}
\end{box-note}

All ranged versions of algorithms that return iterators will return the \cpp{std::ranges::dangling} type when invoked on a temporary range. This would preclude the use case of using \cpp{std::span} to sub-reference a range, which is why the range algorithms have an additional concept of a \cpp{borrowed_range}. Such ranges can be passed in as temporaries since they do not own their elements.