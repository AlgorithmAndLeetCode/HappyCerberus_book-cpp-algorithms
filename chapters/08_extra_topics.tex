\chapter{Extras}

\section{A short arithmetic detour}

The algorithms in the <numeric> header can have sharp edges. This mainly has to do with the way C++ handles mixed numerical types and how template deduction works. For example, the following is an easy mistake to make:

\begin{box-note}
\begin{cppcode}
std::vector<double> data{1.1, 2.2, 3.3, 4.4, 5.5};
auto result = std::accumulate(data.begin(), data.end(), 0);
// result == 15 (actual sum 16.5)
\end{cppcode}
\end{box-note}

Because we are passing a 0 as the initial value of the accumulator, which is a constant of type int, the accumulator ends up as int. Each fold operation then adds an integer and a double, resulting in a floating-point value. However, it immediately stores it in an integer variable, truncating the value.

If you want to avoid this problem, you need to be familiar with literal suffixes:

% TODO: include this information directly in the book, instead of the original links.
\begin{itemize}
    \item integer literal suffixes
    \item floating-point literal suffixes
    \item fixed-size integer literal macros
\end{itemize}

The second problem arises when mixing different integer types, particularly signed and unsigned integers. Again, as long as you do not mix types, you will not run into issues, but remember that the types that matter are the element types, the initial accumulator value and the functor arguments and return type.

\begin{box-note}
\begin{cppcode}
std::vector<unsigned> data{1, std::numeric_limits<unsigned>::max()/2};
auto ok = std::accumulate(data.begin(), data.end(), 0u, 
                          [](auto acc, auto el) { return acc + el; });
// Always OK, matching types.

auto impl = std::accumulate(data.begin(), data.end(), 0, 
                            [](auto acc, auto el) { return acc + el; });
// Implementation defined:
// acc is int
// in acc + el: acc promoted to unsigned, result is unsigned

// if an unsigned value cannot be represented by the target singed variable
// the behavior is implementation defined

auto maybe = std::accumulate(data.begin(), data.end(), 0L, 
                             [](auto acc, auto el) { return acc + el; });
// OK as long as sizeof(long) > sizeof(unsigned)
// acc is long
// in acc + el: el is promoted to long

// if sizeof(long) > sizeof(unsigned), the long can represent all 
// values of unsigned, therefore OK
\end{cppcode}
\end{box-note}

These are arguably very synthetic examples and shouldn’t appear in production codebases.

If you are interested in more details about integer conversions and promotions, there is an excellent cheatsheet from hackingcpp on this topic.

\section{Iterator adapaters}

We have already used the std::back\_inserter many times across the examples in this series, so let's start with inserter adapters.

\subsubsection{back\_inserter, front\_inserter, inserter}

The standard offers three adapters that create instances of back\_insert\_iterator, front\_insert\_iterator and insert\_iterator. When assigned to, these iterators call the corresponding push\_back, push\_front and insert methods on the adapted container.

\begin{box-note}
\begin{cppcode}
std::list<int> data;
auto back = std::back_inserter(data);
auto front = std::front_inserter(data);

*back = 10;
// data == { 10 }
*front = 20;
// data == { 20, 10 }

auto mid = std::inserter(data, std::next(data.begin()));
*mid = 5;
// data == { 20, 5, 10 }
\end{cppcode}
\end{box-note}

All inserter iterators model only output iterator concept, therefore cannot be used with algorithms that require forward or bidirectional iterators.

\subsubsection{istream\_iterator, ostream\_iterator}

These two adapters provide iteration over istream and ostream, respectively.

The istream\_iterator will read a value by calling the corresponding operator>> when incremented.

\begin{box-note}
\begin{cppcode}
std::stringstream input("10 20 30 40 50");
auto in = std::istream_iterator<int>(input);

int i = *in; // i == 10
++in;
int j = *in; // j == 20
\end{cppcode}
\end{box-note}

The ostream\_iterator will write a value by calling the corresponding operator<< when assigned to.

\begin{box-note}
\begin{cppcode}
std::stringstream output;
auto out = std::ostream_iterator<int>(output, ", ");

*out = 10;
*out = 20;
// output == "10, 20,"
\end{cppcode}
\end{box-note}

The additional parameter for ostream\_iterator specifies the divider added after each element.

The istream\_iterator models an input iterator, and the ostream\_iterator models an output iterator.

\subsubsection{move\_iterator, make\_move\_iterator}

The move\_iterator adapter inherits the iterator category from the adapted iterator but will return an rvalue when dereferenced, thus enabling move semantics.

\begin{box-note}
\begin{cppcode}
std::vector<std::string> data = { "a", "b", "c" };
auto it = std::make_move_iterator(data.begin());

std::string str(*it); // move construction
++it;
str = *it; // move assignment
// data == { ?, ?, "c" }
// str == "b"
\end{cppcode}
\end{box-note}

\subsubsection{reverse\_iterator}

We can use the reverse\_iterator adapter for bidirectional iterators to obtain the iterator for the opposite direction.

\begin{box-note}
\begin{cppcode}
std::vector<int> data = { 1, 2, 3 };
auto mid = std::reverse_iterator(data.end());

int i = *mid; // i == 3
++mid;
int j = *mid; // j == 2
\end{cppcode}
\end{box-note}

Note that the reversed iterator maps to the previous element in the source direction. Therefore, when reversing begin or rbegin, the result is rend and end, respectively.

\subsubsection{counted\_iterator}

When working with non-random-access ranges, it is convenient to specify the range using a begin iterator coupled with the number of elements. The counted\_iterator brings this feature to C++20 ranges without the need for specialised algorithm variants like std::fill\_n.

\begin{box-note}
\begin{cppcode}
std::list<int> data = { 1, 2, 3, 4, 5 };

std::ranges::fill(std::counted_iterator(data.begin(), 3), std::default_sentinel, 9);
// data == { 9, 9, 9, 4, 5 }
\end{cppcode}
\end{box-note}

The counted\_iterator keeps track of the number of elements, and when the counter reaches zero, the counted\_iterator will compare equal to std::default\_sentinel, signifying the end of the range.