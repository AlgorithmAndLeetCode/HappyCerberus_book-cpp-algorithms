\section{Set operations}

The group of set algorithms simulates different set operations on two sorted ranges.

\subsection{\texorpdfstring{\cpp{std::set_difference}}{\texttt{std::set\_difference}}}

The \cpp{std::set_difference} algorithm produces a range containing elements present in the first range but not in the second range.

\cppversions{\texttt{set\_difference}}{\CC98}{\CC20}{\CC17}{\CC20}
\constraints{\texttt{(input\_range, input\_range) -> output\_iterator}}{\texttt{(forward\_range, forward\_range) -> forward\_iterator}}{\texttt{operator<}}{\texttt{strict\_weak\_ordering}}

For equivalent elements, where the first range contains $M$ such elements and the second range contains $N$ such elements, the result will contain the last \cpp{std::max(M-N, 0)} such elements from the first range.

\begin{box-note}
\footnotesize Example of using \cpp{std::set_difference}.
\tcblower
\cppfile{code_examples/algorithms/set_difference_code.h}
\end{box-note}

\begin{box-note}
\footnotesize Example demonstrating \cpp{std::set_difference} behaviour when equivalent elements are present.
\tcblower
\cppfile{code_examples/algorithms/set_difference_equal_code.h}
\end{box-note}

\subsection{\texorpdfstring{\cpp{std::set_symmetric_difference}}{\texttt{std::set\_symmetric\_difference}}}

The \cpp{std::set_symetric_difference} produces a range containing elements only present in one of the ranges, but not both.

\cppversions{\texttt{set\_sym\dots diff\dots}}{\CC98}{\CC20}{\CC17}{\CC20}
\constraints{\texttt{(input\_range, input\_range) -> output\_iterator}}{\texttt{(forward\_range, forward\_range) -> forward\_iterator}}{\texttt{operator<}}{\texttt{strict\_weak\_ordering}}

For equivalent elements, where the first range contains $M$ such elements and the second range contains $N$ such elements, the result will contain the last \cpp{std::abs(M-N)} such elements from the corresponding range. That is, if $M>N$, $M-N$ elements will be copied from the first range; otherwise, $N-M$ elements will be copied from the second range.

\begin{box-note}
\footnotesize Example of using \cpp{std::set_symmetric_difference}.
\tcblower
\cppfile{code_examples/algorithms/set_symmetric_difference_code.h}
\end{box-note}

\begin{box-note}
\footnotesize Example demonstrating \cpp{std::set_symmetric_difference} behaviour when equivalent elements are present.
\tcblower
\cppfile{code_examples/algorithms/set_symmetric_difference_equal_code.h}
\end{box-note}

\subsection{set\_union, set\_intersection}

The last group of algorithms that require sorted ranges are set operations.



\cppversions{\texttt{set\_union}}{\CC98}{\CC20}{\CC17}{\CC20}
\cppversions{\texttt{set\_intersection}}{\CC98}{\CC20}{\CC17}{\CC20}

All the set operations work in the same way, processing two input ranges and copying results into the result range, with the following semantics:

\begin{description}
   \item[\texttt{set\_difference}] elements that are present in the first range, but not the second
   \item[\texttt{set\_symmetric\_difference}] elements that are present in only one of the ranges, but not both
   \item[\texttt{set\_union}] elements that are present in either of the ranges
   \item[\texttt{set\_intersection}] elements that are present in both ranges
\end{description}

We also need to talk about the behaviour for equal elements. If we have m such elements in the first range and n such elements in the second range, the resulting range will contain:


\begin{description}
   \item[\texttt{set\_difference}] $max(m-n,0)$ elements
   \item[\texttt{set\_symmetric\_difference}] $abs(m-n)$, 
   \item[\texttt{set\_union}] $max(m,n)$, first $m$ elements will be copied from the first range, followed by $max(n-m,0)$ elements from the second range
   \item[\texttt{set\_intersection}] $min(m,n)$, elements will be copied from the first range
\end{description}

\begin{box-note}
\begin{cppcode}
std::vector<int> data1{1, 2, 5};
std::vector<int> data2{2, 4, 6};

std::vector<int> difference;
std::ranges::set_difference(data1, data2, std::back_inserter(difference));
// difference == {1, 5}

std::vector<int> symmetric;
std::ranges::set_symmetric_difference(data1, data2, std::back_inserter(symmetric));
// symmetric == {1, 4, 5, 6}

std::vector<int> set_union;
std::ranges::set_union(data1, data2, std::back_inserter(set_union));
// set_union == {1, 2, 4, 5, 6}

std::vector<int> intersection;
std::ranges::set_intersection(data1, data2, std::back_inserter(intersection));
// intersection == {2}
\end{cppcode}
\end{box-note}