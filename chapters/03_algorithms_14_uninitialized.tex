\section{Uninitialized memory algorithms}

The uninitialized memory algorithms are a group of relatively low-level algorithms designed to help when implementing manual memory management. These algorithms provide the functionality to transactionally construct, copy, move and destroy sequences of elements on top of raw memory.

% TODO: note and link the advanced chapter on manual memory management.

The counted variants of the algorithms are not listed in this section\footnote{The names of these algorithms are particularly long and obnoxious.}. However, note that all algorithms in this section that operate on ranges have a counted variant, where the range is specified using an iterator and number of elements. These variants are used in some of the examples to demonstrate.

\subsection{\texorpdfstring{\cpp{std::construct_at}, \cpp{std::destroy_at}}{\texttt{std::construct\_at}, \texttt{std::destroy\_at}}}

The \cpp{std::construct_at} and \cpp{std::destroy_at} algorithms will construct/destroy a single element at a given address. If additional arguments are specified, \cpp{std::construct_at} will forward these to the objects’ constructor.

\cppversions{\texttt{construct\_at}}{\CC20}{\CC20}{N/A}{\CC20}
\cppversions{\texttt{destroy\_at}}{\CC17}{\CC20}{N/A}{\CC20}

\begin{box-note}
\footnotesize Example of using \cpp{std::create_at} to create a \cpp{std::string} object using the arguments eight and ‘X’, which results in a string filled with eight copies of the X character.
\tcblower
\cppfile{code_examples/algorithms/create_at_code.h}
\end{box-note}

\subsection{\texorpdfstring{\cpp{std::uninitialized_default_construct},\newline\cpp{std::uninitialized_value_construct},\newline\cpp{std::uninitialized_fill}, \newline\cpp{std::destroy}}{\texttt{std::uninitialized\_default\_construct},\newline\texttt{std::uninitialized_value_construct},\newline\texttt{std::uninitialized_fill}, \newline\texttt{std::destroy}}}

The three uninitialized algorithms cover the default initialization, value initialization and copy initialization of elements. The \cpp{std::destroy} algorithm provides the destruction of elements without deallocating the underlying memory.

\cppversions{\texttt{un\dots default\_construct}}{\CC17}{N/A}{\CC17}{\CC20}
\cppversions{\texttt{un\dots value\_construct}}{\CC17}{N/A}{\CC17}{\CC20}
\cppversions{\texttt{uninitialized\_fill}}{\CC98}{N/A}{\CC17}{\CC20}
\cppversions{\texttt{destroy}}{\CC17}{N/A}{\CC17}{\CC20}

\constraints{\texttt{forward\_range\newline forward\_iterator (counted)}}{}{}{} % Cleanup the counted variant situation

\begin{box-note}
\footnotesize 
Example demonstrating the use of the counted variants of the uninitialized construction algorithm and the \cpp{std::destroy_n} algorithms. Note that for \cpp{std::string}, there is no difference between default and value construction.
\tcblower
\cppfile{code_examples/algorithms/uninitialized_constr_code.h}
\end{box-note}

\subsection{\texorpdfstring{\cpp{std::uninitialized_copy}, \cpp{std::uninitalized_move}}{\texttt{std::uninitialized\_copy}, \texttt{std::uninitalized\_move}}}

The \cpp{std::unitialized_copy} and \cpp{std::unitialized_move} algorithms follow the behaviour of \cpp{std::copy} and \cpp{std::move} algorithms with the distinction that the destination range is uninitialized memory.

\cppversions{\texttt{uninitialized\_copy}}{\CC98}{N/A}{\CC17}{\CC20}
\cppversions{\texttt{uninitialized\_move}}{\CC17}{N/A}{\CC17}{\CC20}

\constraints{\texttt{input\_range -> forward\_iterator\newline input\_iterator -> forward\_iterator (counted)}}{\texttt{forward\_range -> forward\_iterator\newline forward\_iterator -> forward\_iterator (counted)}}{}{}

\begin{box-note}
\footnotesize Example of using \cpp{std::uninitialized_copy} and \cpp{std::uninitialized_move}.
\tcblower
\cppfile{code_examples/algorithms/uninitialized_copy_code.h}
\end{box-note}

\subsubsection{Transactional behaviour}

The main benefit of using the uninitialized memory algorithms is that they correctly handle transactional behaviour. Transactionality is important in cases where the constructor of an object can throw. If one of the objects fails to construct, the algorithms will correctly roll back by destructing already constructed objects.

\begin{box-note}
\footnotesize Example demonstrating the roll-back behaviour of \cpp{std::uninitialized_default_construct_n} when the third invocation of the constructor throws. Note that the exception is re-thrown after the partial work is rolled back.
\tcblower
\cppfile{code_examples/algorithms/transactional_code.h}
\end{box-note}
