\section{Search and compare algorithms}

The search and compare category provides straightforward linear (when compared against a single value) and quadratic (when compared against a range) complexity algorithms.

\subsection{\texorpdfstring{\cpp{std::find}, \cpp{std::find_if}, \cpp{std::find_if_not}}{\texttt{std::find}, \texttt{std::find\_if}, \texttt{std::find\_if\_not}}}
\index{\cpp{std::find}}
\index{\cpp{std::find_if}}
\index{\cpp{std::find_if_not}}

The std::find algorithm provides a basic linear search. The standard provides three variants, one searching by value and two variants using a predicate.

\cppversions{\texttt{find, find\_if}}{\CC98}{\CC20}{\CC17}{\CC20}
\cppversions{\texttt{find\_if\_not}}{\CC11}{\CC20}{\CC17}{\CC20}

\constraints{\texttt{input\_range}}{\texttt{forward\_range}}{\texttt{operator==} (\texttt{find})}{\texttt{unary\_predicate}}

A typical example of searching by value is finding delimiters:

\begin{box-note}
\footnotesize Example of utilizing \cpp{std::find} to find delimiters in a string.
\tcblower
\cppfile{code_examples/algorithms/find_code.h}
\end{box-note}

If we want to search for categories of elements, we can use \cpp{std::find_if} and \cpp{std::find_if_not} since these two variants search using a predicate.

\begin{box-note}
\footnotesize Example of utilizing \cpp{std::find_if_not} to find leading and trailing whitespace.
\tcblower
\cppfile{code_examples/algorithms/find_if_code.h}
\end{box-note}

\subsection{\texorpdfstring{\cpp{std::adjacent_find}}{\texttt{std::adjacent\_find}}}
\index{\cpp{std::adjacent_find}}

The \cpp{std::adjacent_find} is a binary find algorithm that searches for pairs of adjacent elements in a single range.

\cppversions{\texttt{adjacent\_find}}{\CC98}{\CC20}{\CC17}{\CC20}
\constraints{\texttt{forward\_range}}{\texttt{forward\_range}}{\texttt{operator==}}{\texttt{binary\_predicate}}

If the algorithm finds a pair of elements, it will return an iterator to the first of the two elements (end iterator otherwise).

\begin{box-note}
\footnotesize Example of using \cpp{std::adjacent_find} to find the first pair of equal elements and the first pair of elements that sum up to more than ten.
\tcblower
\cppfile{code_examples/algorithms/adjacent_find_code.h}
\end{box-note}

\subsection{\texorpdfstring{\cpp{std::search_n}}{\texttt{std::search\_n}}}
\index{\cpp{std::search_n}}

The \cpp{std::search_n} algorithm searches for n instances of the given value.

\cppversions{\texttt{search\_n}}{\CC98}{\CC20}{\CC17}{\CC20}
\constraints{\texttt{forward\_range}}{\texttt{forward\_range}}{\texttt{operator==}}{\texttt{binary\_predicate}}

The interface to \cpp{std::search_n} can be a bit confusing. The algorithm accepts the number of instances and the value to search for as two consecutive arguments, followed by an optional custom comparator function.

\begin{box-note}
\footnotesize Example of using \cpp{std::search_n} to find two consecutive elements equal to $3$, three elements equal to $3$ (in modulo $5$ arithmetic) and finally, two elements equal to $0$.
\tcblower
\cppfile{code_examples/algorithms/search_n_code.h}
\end{box-note}

Note that \cpp{std::search_n} is one exception to the \cpp{_n} naming scheme.

\subsection{\texorpdfstring{\cpp{std::find_first_of}}{\texttt{std::find\_first\_of}}}
\index{\cpp{std::find_first_of}}

Using \cpp{std::find_if}, we can easily search for a category of elements. However, sometimes it is more convenient to list the elements we are looking for exhaustively.

\cppversions{\texttt{find\_first\_of}}{\CC98}{\CC20}{\CC17}{\CC20}
\constraints{\texttt{(input\_range, forward\_range)}}{\texttt{(forward\_range, forward\_range)}}{\texttt{operator==}}{\texttt{binary\_predicate}}

Note that we are shifting from linear search to $O(m*n)$ time complexity since, for each element of the first range, we need to compare it to all elements in the second range (worst case).

\begin{box-note}
\footnotesize Example of using \cpp{std::find_first_of}.
\tcblower
\cppfile{code_examples/algorithms/find_first_of_code.h}
\end{box-note}

\subsection{\texorpdfstring{\cpp{std::search}, \cpp{std::find_end}}{\texttt{std::search}, \texttt{std::find\_end}}}
\index{\cpp{std::search}}
\index{\cpp{std::find_end}}

Both \cpp{std::search} and \cpp{std::find_end} algorithms search for a sub-sequence in a sequence.
The \cpp{std::search} algorithm will return the first instance, and \cpp{std::find_end} will return the last.

\cppversions{\texttt{search, find\_end}}{\CC98}{\CC20}{\CC17}{\CC20}
\constraints{\texttt{(forward\_range, forward\_range)}}{\texttt{(forward\_range, forward\_range)}}{\texttt{operator==}}{\texttt{binary\_predicate}}

\begin{box-note}
\footnotesize Example of using \cpp{std::search} and \cpp{std::find_end}.
\tcblower
\cppfile{code_examples/algorithms/search_code.h}
\end{box-note}

\subsubsection{Searchers}

Since \CC17, we also can specify custom searchers for the search algorithm. Apart from the basic one, the standard implements Boyer-Moore and Boyer-Moore-Horspool string searchers that offer different best-case, worst-case and average complexity.

\begin{box-note}
\footnotesize Example of using \cpp{std::search} with custom searchers.
\tcblower
\cppfile{code_examples/algorithms/searchers_code.h}
\index{\cpp{std::default_searcher}}
\index{\cpp{std::boyer_moore_searcher}}
\index{\cpp{std::boyer_moore_horspool_searcher}}
\end{box-note}

\subsection{\texorpdfstring{\cpp{std::count}, \cpp{std::count_if}}{\texttt{std::count}, \texttt{std::count\_if}}}
\index{\cpp{std::count}}
\index{\cpp{std::count_if}}

The \cpp{std::count} and \cpp{std::count_if} algorithms count the number of matching elements.

\cppversions{\texttt{count, count\_if}}{\CC98}{\CC20}{\CC17}{\CC20}
\constraints{\texttt{input\_range}}{\texttt{forward\_range}}{\texttt{operator==}}{\texttt{unary\_predicate}}

The element searched for can be specified using a value (\cpp{std::count}) or a predicate (\cpp{std::count_if}).

\begin{box-note}
\footnotesize Example of using \cpp{std::count} and \cpp{std::count_if}.
\tcblower
\cppfile{code_examples/algorithms/count_code.h}
\end{box-note}

\subsection{\texorpdfstring{\cpp{std::equal}, \cpp{std::mismatch}}{\texttt{std::equal}, \texttt{std::mismatch}}}
\index{\cpp{std::equal}}
\index{\cpp{std::mismatch}}

The \cpp{std::equal} algorithm provides an equality comparison for ranges.

\cppversions{\texttt{equal}}{\CC98}{\CC20}{\CC17}{\CC20}
\constraints{\texttt{(input\_range, input\_iterator)}}{\texttt{(forward\_range, forward\_iterator)}}{\texttt{operator==}}{\texttt{binary\_predicate}}

\begin{box-note}
\footnotesize Example of using \cpp{std::equal}.
\tcblower
\cppfile{code_examples/algorithms/equal_code.h}
\end{box-note}

The \cpp{std::mismatch} algorithm behaves exactly like \cpp{std::equal}; however, instead of returning a simple boolean, it returns a pair of iterators denoting the mismatched elements.

\cppversions{\texttt{mismatch}}{\CC98}{\CC20}{\CC17}{\CC20}
\constraints{\texttt{(input\_range, input\_iterator)}}{\texttt{(forward\_range, forward\_iterator)}}{\texttt{operator==}}{\texttt{binary\_predicate}}

\begin{box-note}
\footnotesize Example of using \cpp{std::mismatch}.
\tcblower
\cppfile{code_examples/algorithms/mismatch_code.h}
\end{box-note}

On top of the basic variants, both \cpp{std::equal} and \cpp{std::mismatch} offer a version where both ranges are fully specified (since C++14). These versions can detect mismatches beyond the first range's scope.

\constraints{\texttt{(input\_range, input\_range)} since \CC14}{(forward\_range, forward\_range)}{\texttt{operator==}}{\texttt{binary\_predicate}}

\begin{box-note}
\footnotesize Example of detecting a mismatch beyond the scope of the first range.
\tcblower
\cppfile{code_examples/algorithms/equal_with_range_code.h}
\end{box-note}