\section{Swaps}

Before discussing swaps, we need to make a short detour and discuss argument-dependent lookup, an essential aspect of the pre-C++20 version of the std::swap algorithm.

Argument-dependent lookup is relied upon heavily in C++, notably when overloading operators.

\begin{box-note}
\footnotesize Example of argument-dependent lookup for \cpp{operator<<} implemented as a function.
\tcblower
\cppfile{code_examples/theory/adl_code.h}
\end{box-note}

The situation changes slightly when the function is implemented as a friend function. Such a function is a member of the surrounding namespace. However, it is not visible outside of ADL.

\begin{box-note}
\footnotesize Example of argument-dependent lookup for \cpp{operator<<} implemented as a friend function.
\tcblower
\cppfile{code_examples/theory/adl_friend_code.h}
\end{box-note}

The benefit of relying on ADL is that there is no complexity on the caller's site. An unqualified call will invoke the correct overload. Except, this wouldn't be C++ if there weren't an exception to this rule.

If on top of having the ability to specialize, we also want default behaviour as a fallback, the caller now needs to make sure to pull in the default overload to the local scope.

\begin{box-note}
\footnotesize Example of argument-dependent lookup with a default templated version of the function.
\tcblower
\cppfile{code_examples/theory/adl_default_code.h}
\end{box-note}

\subsection{\texorpdfstring{\cpp{std::swap}}{\texttt{std::swap}}}

The non-range version of \cpp{std::swap} will swap the values of the two parameters using a three-step move-swap. Users can provide a more optimized implementation as friend functions on their type.

\cppversions{\texttt{swap}}{\CC98}{\CC20}{N/A}{\CC20}

Correctly calling swap (as mentioned at the beginning of this section) requires pulling the default std::swap version to the local scope.

\begin{box-note}
\footnotesize Example of correctly calling \cpp{std::swap}.
\tcblower
\cppfile{code_examples/algorithms/swap_calling_code.h}
\end{box-note}
\newpage

The C++20 rangified version of swap removes this complexity, and it will:

\begin{itemize}
    \item call the user-provided (found by ADL) overload of swap
    \item if that doesn't exist and the parameters are arrays of the same span, \cpp{std::ranges::swap} will behave as \cpp{std::ranges::swap_ranges}
    \item if the parameters are also not arrays, it will default to a move-swap
\end{itemize}

\begin{box-note}
\footnotesize Example of specializing and calling \cpp{std::ranges::swap}.
\tcblower
\cppfile{code_examples/algorithms/swap_range_code.h}
\end{box-note}

\subsection{\texorpdfstring{\cpp{std::iter_swap}}{\texttt{std::iter\_swap}}}

The \cpp{std::iter_swap} is an indirect swap, swapping values behind two forward iterators.

\cppversions{\texttt{iter\_swap}}{\CC98}{\CC20}{N/A}{\CC20}

\constraints{(\cpp{forward_iterator}, \cpp{forward_iterator}) (non-range)}{}{}{}

\begin{box-note}
\footnotesize Example demonstrating the use \cpp{std::iter_swap} in a generic two-way partition algorithm. The algorithm uses concepts to constrain the acceptable types of its arguments.
\tcblower
\cppfile{code_examples/algorithms/iter_swap_partition_code.h}
\end{box-note}

The range version extends the functionality to other dereferenceable objects.

\begin{box-note}
\footnotesize Example demonstrating the use of range version of \cpp{std::ranges::iter_swap} to swap the values pointed to by two instances of \cpp{std::unique_ptr}.
\tcblower
\cppfile{code_examples/algorithms/iter_swap_unique_ptr_code.h}
\end{box-note}

\subsection{\texorpdfstring{\cpp{std::swap_ranges}}{\texttt{std::swap\_ranges}}}

The \cpp{std::swap_ranges} algorithm exchanges elements between two non-overlapping ranges (potentially from the same container).

\cppversions{\texttt{swap\_ranges}}{\CC98}{\CC20}{\CC17}{\CC20}


\begin{box-note}
\footnotesize Example of swapping the first three elements of an array with the last three elements using \cpp{std::swap_ranges}. Note the reversed order of elements due to the use of \cpp{rbegin}.
\tcblower
\cppfile{code_examples/algorithms/swap_ranges_code.h}
\end{box-note}