\section{Heap data structure}

The standard offers a convenient wrapper for a max-heap data structure through \cpp{std::priority_queue}.
However, when using \cpp{std::priority_queue}, we lose access to the underlying data, which might be inconvenient.

\subsection{\texorpdfstring{\cpp{std::make_heap}, \cpp{std::push_heap}, \cpp{std::pop_heap}}{\texttt{std::make\_heap}, \texttt{std::push\_heap}, \texttt{std::pop\_heap}}}

A Heap data structure is a binary tree where each element satisfies the heap property: the value at the parent is greater or equal to the value of its children.

When discussing the heap algorithms, we will be referring to an array representation of the heap, where the children of the element at index $i$ are elements at indexes $2i+1$ and $2i+2$.

The valuable property of a heap is that it can be constructed in linear time and then provide logarithmic complexity for extracting the maximum value and inserting new values into the heap.

\cppversions{\texttt{make\_heap}}{\CC98}{\CC20}{N/A}{\CC20}
\cppversions{\texttt{push\_heap, pop\_heap}}{\CC98}{\CC20}{N/A}{\CC20}
\constraints{\texttt{random\_access\_range}}{}{\texttt{operator<}}{\texttt{strict\_weak\_ordering}}

The \cpp{std::make_heap} algorithm reorders elements in the given range such that the elements maintain the max-heap property, that is, the element at index $i$ compares greater or equal to the elements at indexes $2i+1$ and $2i+2$.

\begin{box-note}
\footnotesize Example of using \cpp{std::make_heap} to construct a max-heap and a min-heap (using a custom comparator).
\tcblower
\cppfile{code_examples/algorithms/make_heap_code.h}
\end{box-note}

The \cpp{std::push_heap} and \cpp{std::pop_heap} algorithms simulate push and pop operations for the max-heap data structure. However, because they operate on top of a range, they cannot manipulate the underlying data structure. Therefore, they use the last element of the range as the input/output.

The \cpp{std::push_heap} algorithm will insert the last element of the range into the heap, and \cpp{std::pop_heap} will extract the maximum element to the last position of the range. As previously mentioned, both operations have logarithmic complexity.

\begin{box-note}
\footnotesize Example of using \cpp{std::push_heap} and \cpp{std::pop_heap}.
\tcblower
\cppfile{code_examples/algorithms/push_heap_code.h}
\end{box-note}

\subsection{\texorpdfstring{\cpp{std::sort_heap}}{\texttt{std::sort\_heap}}}

The \cpp{std::sort_heap} will reorder the elements in a heap into a sorted order. Note that this is the same as repeatedly calling \cpp{std::pop_heap}; hence the algorithm has $O(n*logn)$ complexity.

\cppversions{\texttt{sort\_heap}}{\CC98}{\CC20}{N/A}{\CC20}
\constraints{\texttt{random\_access\_range}}{}{\texttt{operator<}}{\texttt{strict\_weak\_ordering}}

\begin{box-note}
\footnotesize Example of using \cpp{std::sort_heap}.
\tcblower
\cppfile{code_examples/algorithms/sort_heap_code.h}
\end{box-note}

\subsection{---}

% mention complexity

For comparison, let’s implement a simple version of the topk algorithm that returns the top k elements from a range (similar to partial\_sort\_copy):

\begin{box-note}
\begin{cppcode}
auto topk_queue(std::input_iterator auto begin, std::sentinel_for<decltype(begin)> auto end, 
                size_t k) {

    using vtype = std::iter_value_t<decltype(begin)>;
    std::priority_queue<vtype, std::vector<vtype>, std::greater<vtype>> pq;

    while (begin != end) {
        pq.push(*begin);
        if (pq.size() > k)
            pq.pop();
        ++begin;
    }

    std::vector<vtype> result(k);
    for (auto &el: result | std::views::reverse) {
        el = std::move(pq.top());
        pq.pop();
    }
    return result;    
}
\end{cppcode}
\end{box-note}

When using the priority queue, we can utilize the simple push() and pop() interface provided (lines 8 and 10). However, extracting all data from the queue is only possible by repeatedly applying pop() until the queue is empty (line 17).

\begin{box-note}
\begin{cppcode}
auto topk(std::input_iterator auto begin, std::sentinel_for<decltype(begin)> auto end, 
          size_t k) {
          
    std::vector<std::iter_value_t<decltype(begin)>> result;
    while (begin != end) {
        result.push_back(*begin);
        std::ranges::push_heap(result, std::greater<>{});
        if (result.size() > k) {
            std::ranges::pop_heap(result, std::greater<>{});
            result.pop_back();
        }
        ++begin;
    }
    
    std::ranges::sort_heap(result, std::greater<>{});
    return result;    
}
\end{cppcode}
\end{box-note}

When using the heap algorithms, we need to manually manage the underlying data structure (lines 6–7 and 9–10). However, we do not need to extract the data, and on top of that, we could omit the final sort\_heap (line 15) if we do not need the top k elements in sorted order.

Our example does not utilize std::make\_heap, as we start from an empty range, which is a valid heap.

\cppversions{\texttt{is\_heap, is\_heap\_until}}{\CC11}{\CC20}{N/A}{\CC20}

\constraints{\texttt{random\_access\_range}}{}{\texttt{operator<}}{\texttt{strict\_weak\_ordering}}

The two test algorithms check the heap invariant. Max heap is a binary tree, where each child is equal to or less than the parent element. The is\_heap variant returns a boolean, where the is\_heap\_until returns an iterator denoting the leading part of the range, which satisfies the max heap invariant.

\begin{box-note}
\begin{cppcode}
std::vector<int> data = {1, 2, 3, 4, 5, 6, 7, 8, 9};

assert(!std::is_heap(data.begin(), data.end()));
assert(std::is_heap_until(data.begin(), data.end()) == std::next(data.begin()));

std::make_heap(data.begin(), data.end());

assert(std::is_heap(data.begin(), data.end()));
assert(std::is_heap_until(data.begin(), data.end()) == data.end());
\end{cppcode}
\end{box-note}

Before applying make\_heap, only the first element is in the correct heap order (parent of 2 is 1, which violates the max heap invariant). Therefore, is\_heap returns false (line 3) and is\_heap\_until returns an iterator to the second element (line 4).

After applying make\_heap, the entire range is in heap order, is\_heap returns true (line 8), and is\_heap\_until returns the end of range iterator (line 9).

Notably, is\_heap(begin, is\_heap\_until(begin, end)) will always return true.